 \documentclass[a4paper, twocolumn]{article}

\usepackage{amsmath, amsthm, amssymb, amsfonts} 
%\usepackage{german}
\usepackage[utf8]{inputenc}
\usepackage{multicol}  
\usepackage{multirow}
%\usepackage{dsfont} 
%\usepackage[rflt]{floatflt}  
\usepackage{graphics}
\usepackage{epsfig} 
\usepackage[pdftex]{hyperref}
\usepackage{moreverb}


% ZuFa Template laden
% Layout by Thorben Bocheneck
\usepackage{tbabstract}


\title{Templates}
\author{Patrick Bänziger}

%Kommandos

\definecolor{lightgrey}{rgb}{0.96,0.96,0.96}
\newcommand{\keyword}[1]{\textbf{\color{black}{#1}}}
\newcommand{\example}[1]{\textbf{\medskip\colorbox{lightgrey}{\color{black}{Beispiel: #1}}}}
\newcommand{\typecomb}{$\blacktriangleright$}

\begin{document}
\maketitle

\section{Examples follow below}

\subsection{Motivation}
As studies have shown, the vast majority ($\approx$ 84\%) of all software projects can be considered unsuccessful. They either are too late, over budget, don't contain all promised features or get outright cancelled. \\

There are two major challenges which need to be addressed in successful software engineering:

\subsubsection*{Complexity}
Complexity in software engineering is introduced by three main causes, namely the:
\begin{itemize}
	\item complexity of the problem domain
	\item complexity of the development process
	\item required flexibility of software
\end{itemize}
Among the aspects of complexity are team fluctuation, multi-version software and conflicting objectives (quality versus price).
imation methods, depending on the situation:\\

$\mathbf{CEAC_1}$: If the current variances (e.g. lower staff performance, higher complexity) will now remain at the same level.
$$CEAC_1 = \frac{BAC}{CPI}$$

$\mathbf{CEAC_2}$: If the past variances will not occur in the future anymore. (We take actual cost for the past and remain at estimates for the future.)
$$CEAC_2 = AC + BAC - EV$$

$\mathbf{CEAC_3}$: If the old estimate was just fundamentally wrong.
$$CEAC_3 = AC + newEstimate$$

\subsubsection{Golden rules}
What remains to be said? There is the question how to calculate the value of an activity that has started.

\begin{enumerate}
	\item Every chunk of earned value should be verified by \textbf{physically examining the product} of the activity.
	\item For unfinished activities, earned value estimates are just a guess! One of the following rules should be applied consistently:
	\begin{itemize}
		\item \textbf{50/50 Rule:} After a tasks starts, it is considered 50\% complete until it is finished.
		\item \textbf{20/80:} After a task starts, it is considered 20\% complete. This is realistic, as the timeconsuming or difficult parts are towards the end of a task.
		\item \textbf{0/100:} The task does not generate any earned value until completion. Very conservative.
	\end{itemize}
\end{enumerate}


\end{document}
